% begin Chapter Introduction

\chapter{Introduction}
\label{chapter-Introduction}

\paragraph{ } The Private Bus Transportation System is a famously contentious topic in Sri Lanka. For the average commuter, complaining about the system is part and parcel of everyday life. As a commuter myself and having used public transportation extensively for more than a decade, I have wondered many a time about the reasons why the bus service is in this state and why people keep complaining about it so very vehemently. The constant complaints are justified, as the bus service is grossly inefficient and lack proper service quality. The inefficiency in the system leads to lost productivity and dissatisfaction by the commuters. Eventually, it is the country that suffers from the lack of an efficient public transportation system for the people.

The initial objective of my project was to pinpoint the existing problems and possibly provide an IT solution for them. During the course of the research and interviews with the numerous people involved with the public transportation sector, it occurred to me that the system has a myriad of things wrong with it. Careful management was the only requirement to overcome some problems, while others required a restructuring of the system, and still others required just an implementation of a solution and regulation of the service properly. Further research was required to identify underlying causes and provide a holistic solution to the problems. As I was passionate about the problem, I was intent on conducting a research to find solutions.

It is in this background that my research into the usage of an IT-based system to solve problems in the local public transportation service was carried out. I did not immediately arrive at a solution, but gradually understood that the problems and the requirements of the stakeholders demanded such a system. Consequently, I attempted to identify and abstract the system's key components so that the over-arching concept can be applied to similar transportation systems in other developing countries.

\section{Background Study: The Case of the \acrshort{wp} \acrshort{rpta}}
\label{section-BackgroundStudy}

\paragraph{ } Bus Passenger Transport is the highest and most widely used mode of public transport in Sri Lanka. The system is divided into Inter-Provincial and Intra-Provincial bus services. The service is operated by privately run (private buses) as well as state-run (\acrshort{sltb}) buses. The Sri Lanka Transport Board manages the \acrshort{sltb} buses and their jurisdiction is island wide regardless of the district or province. However, the organizational structure of private buses is different. All Inter-Provincial Private Buses are governed by the National Transport Commission (\acrshort{ntc}) while the private bus services within each of Sri Lanka's provinces are governed by the Passenger Transport Authorities of the respective provinces. Accordingly, the Western Province Road Passenger Transport Authority (\acrshort{wp} \acrshort{rpta}) handles the governance of the Private Bus Passenger Transport System in the Western Province. Please refer to Figure~\ref{image-busTransportSystemStructure} for an illustration of the Structure.

\begin {figure} [H]
\centering
\includegraphics [scale=0.7] {busTransportSystemStructure}
\caption [Structure of the Bus Transport System in Sri Lanka] {The Structure of the Bus Transport System in Sri Lanka}
\label {image-busTransportSystemStructure}
\end {figure}

The major difference between the system in use in Sri Lanka and other countries is that the owners and operators of the buses are independent contractors and not the Central Government or the Provincial Authority. This has led to numerous problems and we shall discuss them in detail during the course of this document.

\subsection{Statistics of the Bus Service in Sri Lanka}

Let us take a look at some statistics involved with the bus service in Sri Lanka to fully grasp the magnitude of the system. According to the Draft National Policy on Transport in Sri Lanka \cite{MinistryofTransport2008}, public transport accounts for nearly 73\% of the total motorized passenger transport in the country. It also serves as the only means of transport for the majority of the population.

Of this, bus transportation accounts for nearly 68\% (a 93\% share in the public transport sphere), while rail transport accounts for the remaining 5\% (a 7\% share in the public transport sphere) \cite{MinistryofTransport2008}.

Within the bus transportation system, state-run bus services account for a 23\% share (about one-third of the share of bus transport) while private operators have a share of 45\% (a two-thirds share) provided by small-scale operators.

\begin {figure} [H]
\centering
\includegraphics [scale=0.7] {totalMotorTransportPieChart}
\caption [Total Motorized Passenger Transport in SL] {Detailed View of Total Motorized Passenger Transport in Sri Lanka. Source: \cite{MinistryofTransport2008}}
\label {image-totalMotorTransportPieChart}
\end {figure}

Taking a look at the state-run bus service, data obtained from the Sri Lanka Transport Board show that approximately 2.5 million passengers island wide commute daily on close to 4500 \acrshort{sltb} buses \cite{SriLankaTransportBoard2010}. These buses travel an average of 2000km a day \cite{SriLankaTransportBoard2012}. 

Considering the buses operated by private bus operators, it is estimated that 10 million commuters travel daily on approximately 18,000 private buses currently in use in the country \cite{Silva2010}. Data gathered from the \acrshort{wp} \acrshort{rpta} show that there are around 7000 private buses servicing the Western Province and its' routes which is close to 450 in number. 

Looking at these statistics, we clearly see that the buses run by the private bus operators are far more in number compared to the state-run buses. It also shows that the majority of the commuters rely on the buses operated by private bus operators. These buses function as small independent operators and in 30 years of service operation, not even a single major private bus operator has emerged. The private bus "cartel" is highly unionized and dictates terms to the commuters as well as the government on a regular basis \cite{AdaDerana2012}. Bus strikes are very common and when they do happen, the commuters are placed in great discomfort \cite{Samarajiva2012, ColomboPage2012}.

In contrast, the state run bus service has a governing body, the \acrshort{sltb}, but it has been in steady decline since the 1970s owing to mismanagement and cost overruns \cite{AnswersDotCom2012}. It has now become a money-sucking state entity and continues to waste the tax payer's money with no solution on the horizon \cite{LBO2011, Sirimanne2013}. The number of \acrshort{sltb} buses in daily operation is well below the number of private buses and because of this, the state-run bus service is unreliable. Presently, it is but a teardrop in a sea of public transport dominated by the private bus operators. For example, as pointed out previously, there are more private buses in the western province (around 7000) than there are \acrshort{sltb} buses in the entire island (around 4500).Given that the average commuter cannot and does not wait for hours on end till an \acrshort{sltb} bus arrives, they have no other option but to use the private bus service. \cite{Wijayapala2012, Azwer2012}

\subsection{Impact of system on the public}

\paragraph{} As mentioned earlier, the Private Bus service is the largest and most widely used Passenger Transport system not only in the Western Province but also in Sri Lanka. However, commuters are constantly dissatisfied with the service provided and there seems to be no alternative. The railway system has its own problems and inefficiencies and a solution to that demands separate research.

According to World Bank statistics, Sri Lanka's population currently stands at 20.869 million people \cite{WorldBank2013}. Being the most densely populated, the Western Province has a population of 5.8 million people \cite{DepartmentofCensusandStatistics2012}, which is 27.8\% of the total population. This means that more than a quarter of the people in Sri Lanka live and commute in the Western Province. Therefore, it is clearly evident that an improvement in the level of service is needed seeing as it, directly or indirectly, affects more than a quarter of the country's population.

As mentioned previously, there are around 7000 private buses servicing the Western Province. To put that into perspective, the \acrshort{sltb} only has a cadre of 4500 buses island-wide. This means that there are more private buses in the Western Province than there are \acrshort{sltb} buses in the entire country. This shows that it is a large transportation system that affects more than a quarter of the population in the country on a daily basis.

Also of importance to note is that the number of complaints related to the Inter-Provincial Private Bus Service has doubled this year compared to last year. Chairman of the \acrshort{ntc} is reported as saying that over 40 complaints are received per day and this number is increasing \cite{Wickremasekara2012, Range2012}. This means that the current strategy of scheduling buses is clearly not working properly which is leading to increased commuter dissatisfaction and complaints.

Therefore, research into the private bus passenger transport system is essential. It is a large system that affects a large amount of people on a daily basis and the service needs to improve significantly in order for this country to be more productive, the people to be satisfied and for the country to be more attractive to tourists.



\section{Existing problems and their possible causes}
\label{section-ExistingProblems}

\paragraph{} Research observations made about the bus service as well as information gathered through discussions with regular commuters of the bus service and the schedulers at the \acrshort{wp} \acrshort{rpta} \& the \acrshort{ntc} point to several issues that are highlighted frequently . Let us look at these in detail \cite{Wickremasekara2012, Range2012, Mahesh2013a, Theja2013a, Mahesh2013b, Navaratne2013a, Navaratne2013b, Bandara2013, DeSilva2013, Senevirathne2013, Pallegoda2013}.

\subsection{Loitering of buses at bus stops}

\paragraph{} This refers to the idle time the bus spends at a bus stop after passengers have alighted and/or boarded the bus. Loitering of buses is discussed in more detail in Section~\ref{loitering}. On inquiry, the schedulers at the \acrshort{wp} \acrshort{rpta} confirmed that this was a major problem for them. Commuters also frequently complain about the loitering of buses (this occurs even at peak times) but nothing is done about it. Later on in this document we will see how this action hinders the proper operation of the schedules and the service.

\subsection{Overcrowding of buses}

\paragraph{} The overcrowding of buses is another common complaint prevalent with the commuters. Overcrowding is related to the previously discussed point of loitering as one action results in the other. The bus operators purposefully do this in order to increases their daily income. This is a direct result of private bus operators being compensated by their respective owners solely based on the revenue they generate daily. This issue is not one that affects the operators of \acrshort{sltb} buses as they are paid a wage regardless of how much revenue they generate from their trips daily. This points to a larger problem with the structure of the public bus transport system in sri lanka and needs to be addressed by the government and relevant authorities.

\subsection{Slowness of the buses, unreliability of the bus schedule}

\paragraph{} The speed of the buses and the reliability of the schedule is another common gripe among the commuters. At times the buses are too slow, and at other times they go too fast endangering the lives of the passengers in the bus and pedestrians on the road. Commuters also frequently complain about the reliability of a bus schedule. A common remark among commuters is that they sometimes find 2 (or even 3) buses of the same bus route arriving at a bus stop at the same time and no buses after that for quite some time. Traffic might play a role in this but most often this is the result of bus operators not following the schedule properly.

\subsection{Overcharging of fares \& non-issuance of tickets}

\paragraph{} Tickets and fares have always been a source of complaint for the commuters. They frequently point out that they do not receive a proper ticket for the journey they paid for. Although bus operators have been instructed to use a modern Electronic Ticketing Machine, most buses still don't have one and some buses that do don't use it all the time.

Commuters also show their dissatisfaction frequently at the bus conductor for not providing the balance money when they pay for their journey. This amounts to overcharging the fare and although people complain about it, bus operators keep doing it. Added to that the fact that operators frequently go on strike demanding bus fares to be increased, the commuters have a less than accommodating attitude towards bus operators.

\subsection{Unprofessional \& discourteous service}

\paragraph{} This is a common gripe among the commuters. They generally complain about the cleanliness and general presentability of the drivers and the conductors. Some commuters also point out that during most times the operators do not wear the designated uniform.

Unpleasant or discourteous service is another common complaint that the commuters point out. Being rude to passengers, using foul language and disrespecting passengers are all complained about frequently.

\subsection{Reckless driving \& neglect of road rules}

\paragraph{} Commuters and motorists alike complaint about this fact. Private bus operators are notorious to disregard  road rules and drive recklessly. Although the police regularly crack down on errant drivers, this practice continues. 

The bus operators are also well known for picking up and dropping off commuters at places where there is no designated bus stop. This comes under neglecting road rules and is a nuisance to motorists not forgetting the effect it has on the schedule and the overcrowding of the bus.

\subsection{Lack of information on bus transport service}

\paragraph{} A common suggestion that was put forth by the commuters and the schedulers was that an information system would be greatly helpful for the passengers. Currently, there is no system (official or otherwise) that provides information about the bus services and passengers have no way of finding out information except through word-of-mouth and conventional wisdom. This is a major drawback considering the number of people that use buses as their main mode of daily transport.



\section{Current bus scheduling process in sri lanka}
\label{section-CurrentBusSchedulingProcessInSriLanka}

\paragraph{ } Bus scheduling and timetabling in Sri Lanka (both inter and intra-provincial) has been a manual process in the past and it continues to be in the present day. Although IT tools can be used for scheduling and timetabling, the authorities still use an age-old manual process of scheduling buses for the bus routes \cite{Mahesh2013a, Theja2013a, Mahesh2013b}.

Up until the mid 2000's, the bus routes in the Western Province did not have proper bus schedules. The buses were dispatched from the terminals in the order and frequency they arrived. No scientific methodology was used in identifying headways, slack times or schedules. However, following the research efforts from the University of Moratuwa and Mr. Anuradha Piyadasa in particular, bus schedules were implemented, circa 2005. The methodology used in formulating these schedules is explored later in this section. These schedules were then agreed upon and put into circulation in the Western Province bus routes. Although, the formulation of these schedules was done using a software system (developed by Mr. Piyadasa), this system is not being used at the moment. The main steps of the process are illustrated in Figure~\ref{image-currentBusSchedulingProcess}.

\begin{figure}[H]
\centering
\includegraphics [scale=0.7] {currentBusSchedulingProcess}
\caption[Current Bus Scheduling Process in SL]{Current Bus Scheduling Process in Sri Lanka. Source: \cite{Piyadasa2005}}
\label{image-currentBusSchedulingProcess}
\end{figure}

This process of bus scheduling is an extension of the process put forth by Dennis Huisman in 2004 \cite{Piyadasa2005}. The first step of the process is to conduct a survey. This identifies the existing passenger demand, the quality of the service the buses provide on a given route, the requirements of the regulatory body and any special characteristics of a given route. 

\subsection{Collecting survey data}

The surveyors ride the buses at variously chosen days and times (to and from the main terminal), note down the trip times and identify the delays and loitering points of the buses on the given route. They also gauge the passenger demand by noting how crowded the bus gets at different stages of the trip. This is a rough estimate of the demand, as the surveyors do not have access to the exact quantitative data \cite{Mahesh2013a, Theja2013a, Mahesh2013b, Navaratne2013a, Navaratne2013b}.

\subsection{Formulating the timetables}

The next step is the timetable. They formulate this by calculating the average headway for the route. During the formulation of the timetable, the schedulers determine the time the bus service commences and terminates daily for the route. This is done through analyzing the passenger demand data they gather. It is important to correctly identify the time the service commences and terminates daily as it affects the revenue gained by the bus operators and may lead to the operators being unhappy with the timetable. If the service starts too early in the morning or ends too late at night the operators will run at a loss and will not be able to provide a proper service \cite{Mahesh2013a, Theja2013a, Mahesh2013b, Navaratne2013a, Navaratne2013b}.

\subsection{Scheduling of buses and placing the new schedules into circulation}

Next is the scheduling of the buses to the timetable. The scheduling officers schedule the buses to the bus routes manually using their observations, experience and knowledge. Once the scheduling is complete, the crew rosters are formulated using the schedules. Finally, the revised schedule is placed into circulation and used on the bus route \cite{Mahesh2013a, Theja2013a, Mahesh2013b, Navaratne2013a, Navaratne2013b}.

\subsection{The bus schedules currently in operation}

As mentioned previously, until the mid-2000's the private buses in the Western Province did not have predefined schedules to operate with. The buses were dispatched as soon as they came in to the terminal and the operators did not have predefined working hours and trips to complete for each working day \cite{Piyadasa2005}. However, thanks to research efforts by the University of Moratuwa, almost all of the 450+ routes in the Western Province now have working timetables. Despite these timetables being implemented and in operation, they are not being properly adhered to and the quality of the bus service is still well below what is needed.

The bus schedules that are currently in operation take into account both the Economic and Financial cost of the service to the country, the commuters and the bus operators and optimizes the dispatching of the buses via optimal headway manipulation \cite{Piyadasa2005}. After doing numerous mathematical calculations, the methodology identifies the optimal average headway for the route. After identifying the headway, the available buses are scheduled to fill up the timetable as optimally as possible.

As the crew of a particular bus only operates that bus, the steps of vehicle scheduling and crew scheduling could be and should be carried out together. This is known as Integrated Vehicle and Crew Scheduling and literature to support this has been mentioned in ~\ref{chapter-LitReview} of this thesis document.

\subsection{Usage of an information system in the planning process}

\paragraph{ } A software system was developed to be used to calculate the headways. However, the system is not being used by the \acrshort{wp} \acrshort{rpta} to which it was developed. Instead they use a manual method of determining the headways. The \acrshort{ntc} however, uses the system to schedule the inter-provincial bus system \cite{Piyadasa2013}. This system that was developed still has its shortcomings. It only caters to the timetabling part of the planning and feedback processes of the schedulers. This means that it only enables a very reactive approach to solve the problems existing in the system. However, a proactive approach is what is required so that timetables are much more flexible and information is made available for all the stakeholders concerned.

\subsection{Other issues in the present process and system}

\subsubsection{Too many buses} 

\paragraph{} On inquiry from the personnel at the Scheduling Unit of the \acrshort{wp} \acrshort{rpta}, they said that a common observation was that there is a plethora of buses for many of the bus routes in the province, which leads to scheduling difficulties for the Authority. This is due to previous regimes simply doing surveys on the bus service and adding additional buses to the roads, regardless of the requirement. Bus route permits have been issued in mass numbers as a stop-gap solution to the ailing bus service without accounting for the effect it will have on the system as well as the traffic situation. This leads to the revenue gained by each individual bus gradually decreasing (The number of passengers aka the demand stays fairly constant while the number of buses increases which means there are more buses to share in the same revenue. This means that the revenue gained by each individual bus decreases). This has and continues to hinder any efforts for proper reform, restructuring and reengineering of the private bus service in the country.

\subsubsection{Bus operators circumventing the current schedules} 

\paragraph{} The timetables that have been implemented currently have research backing them but the bus operators have found numerous ways to circumvent the objectives of these schedules which is to ensure a timely and efficient bus service. This became clearly evident during discussions with the Scheduling Unit of the WP RPTA and from the constant dissatisfaction by the commuters. The bus operators consistently look to maximize their profits with a disregard for the level of service offered to the passenger. The timekeepers and Officers-In-Charge at bus terminals also add to the problem by accepting bribes and neglecting their duties.

Obvious solutions to these problems would be to reduce the trip times in the schedules even more and implement tougher regulations. However, trip times can only be reduced up to a certain point before they become impracticable. The Scheduling Unit is revising the schedules at the moment but it is moving at a snail's pace, not unlike the buses they schedule. 

\subsubsection{Revision process for existing timetables} 

\paragraph{} Also of note is how the revision process is handled. A revision to an existing schedule is not done unless and until sufficient complaints are received from the commuters. This is a very reactive stance to the situation which is the incorrect policy to adopt as passenger demands vary and the service and the timetable needs to adjust accordingly. At present however, once the timetable is formulated and signed off by the higher authorities, it is not changed unless a significant number of people complain about it. The service and timetable needs to be more proactive in order to offer a better service to the public \cite{Mahesh2013a, Theja2013a, Mahesh2013b, Navaratne2013a, Navaratne2013b, Ranjith2013a}.

The complaint process too has its glaring shortfalls. The mechanism for complaints is non-structured and fairly disorganized. The commuters are not made aware of the complaint mechanism (no proper channel of disseminating information to the public by way of a website/mobile app etc) and complaints are not received, documented and acted upon in a proper fashion (At times only the security guard at the scheduling unit is available to answer the phone and take down complaints when they are received) leading to the status quo continuing \cite{Mahesh2013a, Theja2013a, Mahesh2013b, Navaratne2013a, Navaratne2013b, Ranjith2013a}.



\section{Possible Information Systems solution}
\label{section-possibleISSolution}

\paragraph{} So far in this chapter, we have seen how the private bus transportation system is functioning presently and the myriad of problems associated with the service. After meeting and having discussions with the personnel at the scheduling unit of the \acrshort{wp} \acrshort{rpta}, it was clearly evident that an information system that provides the schedulers with information to aid their decision making process was required. This information was not only important for the schedulers but also for the commuters as well as the owners \& operators of the buses. 

Information systems (\acrshort{is}) is the study of complementary networks of hardware and software that people and organizations use to collect, filter, process, create, and distribute data\cite{Denning1999}. When considering information systems, there are various types in use by organisations ranging from Transaction Processing Systems to Executive Information Systems. They serve specific purposes and are designed to provide proper information to specific managerial and organisational hierachical levels. Please refer to Figure for an illustration of the purposes of the various \acrshort{is}'s and how these Information Systems relate to the organisational hierarchy. A brief description of what each system is and what it does is given below.

\begin{figure}[H]
\centering
\includegraphics [scale=0.6] {informationSystemTieredModel}
\caption[Tiered Model of Information Systems]{Tiered Model of Information Systems \cite{Laudon1988}}
\label{image-informationSystemTieredModel}
\end{figure}

% COMPLETE THIS

\subsection{Transaction Processing Systems}

\paragraph{} A transaction processing system collects and stores data about transactions and sometimes controls decisions made as part of a transaction. The transaction is the activity that changes stored data, examples of such an activity would be using a credit card, making a reservation or making a cash withdrawal at an \acrshort{atm} \cite{NSWHSC2013}.

There are two types of transaction processing:

\begin{itemize}
\item Batch transaction processing: this is information that is gathered and stored but not processed immediately i.e. the processing of an invoice or cheques in a banking system.
\item Real time transaction processing: this is a transaction which is processed immediately and the operator has access to on-line database i.e. withdrawal from a bank account, Library loans.
\end{itemize}

\subsection{Management Information Systems}

\paragraph{} Management Information Systems are systems or processes that provide the information necessary to manage an organization effectively \cite{Comptroller1995, TexasAMUni2012}. 

\subsection{Executive Information Systems}

\paragraph{} 

\subsection{Decision Support Systems}
\label{section-DSSIntro}

\paragraph{} Decision Support Systems are defined as a set of related computer programs and the data required to assist with analysis and decision-making within an organization. The emphasis in DSSs are towards the assistance they provide for the decision-making process \cite{Turban2005}.

\subsection{Which Information System is appropriate?}

According to the nature of the problems currently in the Transportation Service, it is evident that an information system that provides the system stakeholders with information to aid their decision making process is what is needed which means a \acrshort{dss} is the solution that seems to be the right fit for the problem.

Implementation of a Management Information System or a standalone system to create timetables for the schedulers does not address all of the issues for all of the stakeholders concerned. Although an \acrshort{mis} does accomplish the outcome of better management for the Schedulers, it does little to manage the customer-facing activities. It also focuses on management of the organization instead of aiding in the decision-making process. Alternatively, a standalone system to merely create timetables and schedules for the bus routes does not solve all of the problems faced by all the stakeholders. 

Therefore, the need for a Decision Support System to aid the various stakeholders in their decision-making processes as well as enabling the proper monitoring and regulation of the service is clearly evident. The next section will provide a brief introduction to what \acrshort{dss}'s are and how they relate to the problem at hand.  Further literature regarding \acrshort{dss}'s are presented in Section~\ref{section-DSS}.

\subsubsection {Why are \acrshort{dss}'s more appropriate?}

The requirements of the stakeholders in the Transportation Service are such that it suits the offering that DSSs provide. A mechanism that aids the Schedulers in their timetabling process workflow while providing Commuters with the necessary information about the Transportation Service that helps them arrive at an informed decision is exactly what is required. Furthermore, the system's requirements go beyond a simple \acrshort{mis} or a \acrshort{tps}.

In a real-world scenario, a typical decision-making process involves the following characteristics \cite{Fedra2000},

\begin {itemize}
\item Multiple actors
\item Conflicting objectives
\item Multiple criteria
\item Plural rationalities
\item Hidden agenda
\end {itemize}

These are consequently exactly what is prevalent in the current Transportation Service in Sri Lanka.  The following components as included in a general DSS architecture \cite{Fedra2000}.

\begin {itemize}
\item Information resources
\item The analytical engine
\item The user interface
\end {itemize}

Therefore, it is clear that a \acrshort{dss} is the most suitable Information Systems solution for this type of problem.



\section{Transportation \acrshort{is}'s of cities in other developing countries}

\paragraph{} During this research, data regarding \acrshort{is}'s related to transportation services of 4 Indian cities, namely Bangalore, Mumbai, Delhi and Chennai, as well as the transportation services in Uganda (mainly Kampala) and the public bus transport service of Buenos Aires in Argentina was uncovered. Sri Lanka, India, Argentina and Uganda are all listed as Developing Countries according to the latest International Statistical Institute data \cite{ISI2013}.

Of the 4 Indian cities under consideration, Bangalore, Mumbai and Delhi had more modern Information Systems for their transport services. The Chennai system looked outdated and lacked proper maintenance. Common functionality that these 4 systems related to the indian cities provided was a Bus Route Finder (i.e.: when origin and destination stops are provided, the system provides the possible routes), Information Portal regarding Bus Routes and Bus Stops and information regarding hiring/renting a bus for private journeys. Furthermore, only the Bangalore and Chennai systems offered the functionality of commuter feedback \cite{BMTC1997, BEST1995, DTC2012, MTC2001}.

In Argentina, the public bus transport system is known as the "Colectivos" and although there is no proper information system for the bus service, private travel organizations have information portals regarding the service. One such portal is that which is run by Omnilineas, a licensed travel agency in Buenos Aires with a special focus on incoming tourism and bus travel. Their system provides Bus Route Information as well as information regarding booking tickets for long-distance trips \cite{Omnilineas2013}.



\section{Research Goals \& Objectives}
\label{section-ResearchGoalsAndObjectives}

\paragraph{ } Considering the problems mentioned in Section~\ref{section-ExistingProblems}, 
it is evident that a IS solution is required to improve the present system and help the stakeholders. In Section~\ref{section-possibleISSolution}, a description of what kind of information system would be more suitable was provided.

 Therefore, we can identify the main goal of the research project as the creation of an information systems solution that assists the schedulers of the transport authority while also providing information to the commuters to help their decision-making process. 

The schedulers main business process is the formulation of timetables and schedules and thus addressing it is a core requirement. On the other hand, commuters need to have information about the bus transportation service on demand so that they can decide which routes to take, which buses to avoid etc. These two parties also need to be connected in the process of service feedback so that the schedulers are provided with an indication of the level of service provided by the bus operators. This would lead to a more efficient passenger transport service and will improve commuter satisfaction and productivity.

This research also identifies the key aspects of the information system that will be used in this project so that it can be applied to other developing countries with similar environments. This will help other developing countries improve their processes and systems so that a better service will be provided to commuters in the public transportation system.



\section{Research Scope \& Limitations}
\label{section-ResearchScope}

\paragraph{ } The research focuses on finding an \acrshort{is} solution to solve the problems of timetabling, scheduling and information availability. Improving the Schedules and providing a method for commuters to give feedback is what the research project strives to achieve. By achieving this, the research hopes that the private bus transport service is improved and made more efficient and reliable.

The research analyzes how an \acrshort{is} could be applied to the domain of public transportation in a developing nation. The project then attempts to evaluate the prototype system by testing it on a group of users. Data was collected via a quantitative survey for the Commuters as well as a qualitative analysis for the Bus Schedulers in the \acrshort{wp} \acrshort{rpta}.

The data gathering and automated data collection aspects of the schedulers business process is not within the scope of this research. This project assumes that the data is available already and its integrity and validity is intact. This assumption has been made due to time and resource constraints in this 12-month undergraduate research project.

The research also does not attempt to provide or test any hypotheses related to the application of an Expert System-like solution. The project will limit itself to understanding the IS requirements of the domain and attempting to provide a solution to the existing problems that will cater to the requirements of the stakeholders.



\section{Outline of Thesis}
\label{section-OutlineOfThesis}

Beyond this point, the thesis document will be structured as follows. Chapter~\ref{chapter-LitReview} will provide a comprehensive review of the literature regarding the planning process of a typical public transportation system. The Chapter also provides a brief look at how Automated Data Collection helps in the planning process of public transportation systems. The Chapter ends with an overview of the literature related to Decision Support Systems and gives the reader a look at the concept of \acrshort{dss}'s. The Chapter will also touch on the history and evolution of \acrshort{dss}'s and possible future developments of the concept.

Chapter~\ref{chapter-ResearchMethodology} will discuss the research methodology followed in this project. It details the steps taken in the research process and identifies possible alternatives for the existing problems. The Chapter will also provide information on a Pilot Route Survey that was conducted to gauge the requirements of the system and the shortcomings of the transportation service. A stadholder analysis of the public transportation system is also provided in the chapter.

The details of Gaman, a solution prototype that was built for this project will be provided in Chapter~\ref{chapter-SolutionPrototype}. the Chapter will detail all the functionality as well as give the reader an overview of the entities in the prototype. It will end by providing some UI screenshots of the prototype system and detailing some test cases used to test the prototype prior to deployment.

Chapter~\ref{chapter-ResearchEvaluation} will detail the methodology followed in the Evaluation phase of the research. The steps taken to evaluate the concept as well as the evaluation results will be provided in this chapter.

Finally in Chapter~\ref{chapter-ConclusionAndFutureWork}, the conclusions of this research project will be provided and possible future work will be discussed.


