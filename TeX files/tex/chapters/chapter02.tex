% begin Chapter LiteratureReview

\chapter {Literature Review}
\label{chapter-LitReview}

\paragraph{ } In the previous chapter, we discussed and analyzed the background and current state of the private bus service in the Western Province. An introduction to Decision Support Systems and how they relate to the transportation problem was also provided. The chapter also included a thorough explanation of the current bus scheduling process in the WP RPTA. We shall now take an in-depth look at the theoretical aspects of the planning process of public transportation, a more thorough analysis of Decision Support Systems and look into a few transportation systems in other countries.

\section{Planning Process of Public Transportation}

\paragraph{ } The planning and scheduling process of a public transport company generally comprises of 4 main steps, which are \textit{timetabling}, \textit{vehicle scheduling}, \textit{crew scheduling}, and \textit{crew rostering}. The transport mode could be a bus, tram, metro, train or airline. Figure~\ref{image-genericSchedulingProcess} illustrates the flow of the traditional scheduling process. It is carried out as is or the crew scheduling is carried out prior to the vehicle scheduling in certain instances. However, in recent times, the industry has begun adopting an integrated approach of carrying out the vehicle and crew scheduling together. This is also discussed later in the chapter \cite{Huisman2004}.

\begin {figure} [h!]
\centering
\includegraphics {genericSchedulingProcess}
\caption [Scheduling Process of a Public Transport Company] {Scheduling Process of a Public Transport Company. Source: \cite{Huisman2004}}
\label {image-genericSchedulingProcess}
\end {figure}

Decisions about which routes or lines to operate and how frequently, are inputs for the operational planning process. They can be either given by the marketing department of the company or determined by (local, regional or national) authorities. Furthermore, the travel times between various points on the route are assumed to be known. Based on the lines and frequencies, timetables are determined resulting in trips with corresponding start and end locations and times.

The second planning process is vehicle scheduling which consists of assigning vehicles to trips, resulting in a schedule for each vehicle. The vehicles, which are not in use for some time, are parked in a depot. A schedule for a vehicle is split into several vehicle blocks, where a new vehicle block starts at each departure from the depot. On such a block a sequence of tasks can be defined, where each task needs to be assigned to a working period for one crew (a duty) in the crew scheduling process. The feasibility of a duty is dependent on a set of collective agreements and labor rules that refer to sufficient rest time etcetera. Crew scheduling is short term crew planning (one day), i.e. assigning crew duties to tasks on each specific day, while the crew rostering process is long term crew planning (e.g. half a year) for constructing rosters from the crew duties.

\subsection{Timetabling}

\paragraph{ } Timetabling is the process of determining how frequently buses must operate on routes based on passenger demand and the operational plan of the respective authority and generating corresponding start and end locations and times of trips, based on these frequencies. Furthermore travel times between various points in routes, lengths of routes are assumed to be known \cite{Huisman2004}.

The main consideration in the process of timetabling is the output of headways, or the time between consecutive buses. Currently, average optimal headways are calculated in Sri Lanka and used in the scheduling process. An explanation of this process has been given in the previous chapter.

Let us take a look at previous research done into this area. \cite{Newell1971} provides the basic dispatching policy for a transportation route. This work has been the basis for many future work including \cite{Kumarage2007} research paper on formulating an optimal bus dispatching policy under variable demand over time and route length as is the case in Sri Lanka. The paper considered a method that was an extension to \cite{Newell1971}'s Optimal Dispatching Policy to determine a fleet size and dispatching rate based on both the operator's cost and user's cost including the disutility of standing, in order to arrive at a global cost optimum.

\cite{Riano2004} propose a stochastic model for Bus Dispatching that uses a linear programming model so that it is solvable easily and can be used for other modes of mass transit as well. A feature of this model is that it should be applied to modes of mass transit where the frequency is high enough so that users do not need to know the
schedule in advance (such as the one in Sri Lanka) The solution technique is based on a novel Transient Little Law.

Alternatively, \cite{Daganzo2008} describes an adaptive control scheme to mitigate the problem Bus Bunching in a Public Bus Transport system. This is a problem closely related to timetabling and scheduling and thus has been included in this Literature Review. Bus schedules cannot be easily maintained on busy lines with short headways. Experience shows that buses offering this type of service usually arrive irregularly at their stops, often in bunches. Although authorities build slack into their schedules to alleviate this problem, if necessary holding buses at control points to stay on schedule, their attempts often fail because practical amounts of slack cannot prevent large localized disruptions from spreading system-wide. 

The proposed scheme dynamically determines bus holding times at a route’s control points based on real-time headway information. The method requires less slack than the conventional, schedule-based approach to produce headways within a given tolerance. This allows buses to travel faster, reducing in-vehicle passenger delay and increasing bus productivity. One disadvantage of this method when thinking in terms of a Sri Lankan context is that it requires real-time information which cannot be obtained currently in the Bus Transport System in Colombo.

\cite{Xuan2011} look at the timetabling problem and its effect on schedule reliability a little differently. They study several Dynamic Holding Strategies that use the current state of all buses, as well as a virtual schedule. The virtual schedule is introduced whether the system is run with a 
published schedule or not. Through their research, it was found that through a dynamic holding strategy buses can both closely adhere to schedule and maintain regular headways without too much slack. This in turn improves Schedule Reliability and Commercial Speed of the buses.

\cite{Ceder2009} in his paper, put forth a methodology framework with developed algorithms for the derivation of vehicle departure times (timetable) with either even headways or even average passenger loads. The latter would be ideal for a situation like in Sri Lanka where overcrowding is a major complaint among the commuters.

\cite{Qian2013}, in their paper studied the Optimizing Mathematical Model of Bus Departure Interval and its Solution, and then worked out the best Bus Departure Interval (i.e.: the Headway). They established the optimizing mathematical model of bus departure interval, which took crowding cost of passengers into account, including passengers’ on-the-bus time cost, passengers’ crowding cost and bus company cost. The paper used a genetic algorithm to solve the problem. Reasonable bus departure intervals were obtained quickly by this method.

Furthermore, \cite{Fu2003} presented a new transit operating strategy in which service vehicles operate in pairs with the lead vehicle providing an all-stop local service and the following vehicle being allowed to skip some stops as an express service to address the bus dispatching and timetabling problem. The underlying scheduling problem is formulated as a nonlinear integer programming problem with the objective of minimizing the total costs for both operators and passengers. This could possibly be adapted into the Sri Lankan context as a viable alternative to the current scheduling methodology.

Finally, \cite{Sun2008} studied the headway optimization and scheduling combination of Bus Rapid Transit (BRT) vehicles. A model was proposed to minimize passengers' travel costs and vehicles' operation cost, and constraints included passenger volume, time, and frequency. The scheduling combination was composed by Normal, Zone, and Express scheduling. The model was solved by a genetic algorithm of variable-length coding. The result of the numerical case showed that the optimization results can save 69.92\% of the cost. A sensitivity analysis that was carried out showed that, under higher traffic volume or lower speed, the travel cost can be reduced through reasonable scheduling combination. The method was, therefore, proven scientifically and is feasible. A similar scheduling method of using Normal, Zone and Express Scheduling could possibly be implemented in Sri Lanka.

\subsection{Vehicle Scheduling}

\paragraph{ } Vehicle scheduling is defined as the process of assigning vehicles, to a set of predetermined trips, with fixed starting and ending times, while minimizing capital and operational costs. According to \cite{Freling2003}, the main objective of this step in the Scheduling Process is keeping the operational and capital costs of vehicles to a minimum.

The Vehicle Scheduling problem can be thought of through two perspectives. They are the Single-Depot Vehicle Scheduling Problem (SDVSP) and the Multiple-Depot Vehicle Scheduling Problem (MDVSP). As noticeable by their names, the difference is the number of depots that each Vehicle is assigned to. In the former, a given vehicle is assigned to a single depot while the latter situation assumes that a given vehicle is assigned to multiple depots. The SDVSP further assumes that all vehicles are identical and there are no time constraints \cite{Huisman2004}. Of these 2 approaches, only the SDVSP applies to the Sri Lankan context as each bus (and its crew) is issued a permit to ply on one route. Therefore, \cite{Freling2003} defines the Single-Depot Vehicle Scheduling Problem as follows.

“Given a depot at location \textit{d} and \textit{n} trips from locations \textit{b}$_i$ to \textit{e}$_i$, with corresponding fixed times \textit{bt}$_i$ and \textit{et}$_i$ (\textit{i}=1,...,\textit{n}),  and  given  the  travelling  times  between  all  pairs (\textit{d}, \textit{b}$_i$), (\textit{b}$_i$, \textit{e}$_i$), (\textit{e}$_i$, \textit{b}$_j$) and (\textit{e}$_i$, \textit{d}), find a feasible \textit{minimum cost} schedule for the vehicles, such that all trips are covered by a vehicle. Each trip has to be entirely serviced by one vehicle and trips serviced by the same vehicle are linked by \textit{deadheading trips} (\textit{dh}-trips).  These are trips without serving passengers (pairs (\textit{d}, \textit{b}$_i$), (\textit{e}$_i$, \textit{b}$_j$) and (\textit{e}$_i$, \textit{d})), consisting of travel time (vehicle deadheading) and/or \textit{idle time} (vehicle waiting time). A schedule for a vehicle is composed of vehicle \textit{blocks}, where each block is a departure from the depot, the service of a sequence of trips and the return to the depot. The cost function is a combination of vehicle capital (fixed) and/or operational (variable) cost. The capital cost is often such that the number of vehicles will be minimized, while the operational cost is often a combination of vehicle travel and idle time”.

In the Multiple-Depot Vehicle Scheduling Problem (MDVSP), buses are dispatched from several depots and total vehicles costs have to be minimized subject to the following constraints \cite{Huisman2004}

\begin {itemize}
\item Every vehicle is associated with a single depot; 
\item Every trip has to be assigned to exactly one vehicle; 
\item Some trips have to be assigned to vehicles from a certain subset of depots.
\end {itemize}

\subsection {Crew Scheduling}

\paragraph{ } The next step in the traditional scheduling process is the Crew Scheduling.  The problem is defined as follows: “The Crew Scheduling Problem (CSP) deals with assigning tasks to duties such that each task is performed; each duty is feasible with respect to a set of working rules and the total costs of the duties are minimized”. 

According to \cite{Huisman2004}, the only requirement with relation to the feasibility of a duty is its length or the working time in that duty, respectively. In most literature, the CSP is formulated as a set partitioning or covering problem and solved with a column generation approach.

\subsection {Integrated Vehicle and Crew Scheduling}

\paragraph{ } As mentioned previously in this chapter, the industry has begun using Integrated Vehicle and Crew Scheduling in recent times. This is due to logical as well as financial reasons. The airline industry uses this methodology as a crew has to be scheduled along with an airline trip for obvious reasons.

The Integrated Vehicle and Crew Scheduling Problem (Integrated VCSP or IVCSP) is analyzed by \cite{Huisman2004}. Accordingly, the Integrated Vehicle and Crew Scheduling Problem is defined as follows. “Given a set of service requirements or trips within a fixed planning horizon, find a minimum cost schedule for the vehicles and the crews, such that both the vehicle and the crew schedules are feasible and mutually compatible. Each trip has fixed starting and ending times, and the travelling times between all pairs of locations are known. A vehicle schedule is feasible if (1) each trip is assigned to a vehicle, and (2) each vehicle performs a feasible sequence of trips, where a sequence of trips is feasible if it is feasible for a vehicle to execute each pair of consecutive trips in the sequence”.

The author provides a very comprehensive look into the problem and presents a methodology to solve integrations of both the SDVSP and the MDVSP. The Integrated VCSP is also discussed by \cite{Freling2000} and \cite{Wren1997}.

\subsection {Crew Rostering}

The Crew Rostering Problem (CRP) aims at determining an optimal sequencing of a given set of duties into rosters satisfying operational constraints deriving from union contracts and company regulations. In the public bus transport industry, the roster is used to evenly distribute the workload among the crew \cite{Caprara1995}. Further studies were done by \cite{Kharraziha2003} and \cite{Tian2012}. The Integrated Crew Rostering Problem was examined in depth by \cite{Valdes2010} and \cite{Xie2012}.

This literature review looked at the different steps involved in the bus scheduling process. It also reviewed the literature related to each step. This literature is what will act as reference material as we attempt to present a new model of scheduling buses. The next chapter will detail the research methodology and the proposed model for scheduling buses in the Western Province.

\newpage

\section{Decision Support Systems}
\label {section-DSS}

\paragraph{ } Decision Support Systems are defined as a set of related computer programs and the data required to assist with analysis and decision-making within an organization. They are purposely developed to improve decision making of non-structured management problems. Decision Support Systems utilize data, provide an easy-to-use interface, and allow for the decision maker’s own insights hence being interactive, flexible, and adaptable computer-based information systems \cite{Turban2005}. 

Having said that, the term DSS in itself is difficult to define with academics and users differing in their opinion in terms of its preferred use. While academics have perceived DSS as a tool to support the decision making process, DSS users see DSS as a tool to facilitate organizational processes \cite{Keen1980}.

\subsection{History of Decision Support Systems}

\paragraph{ } The concept of a Decision Support System has been around since the 1950's but only recently has it increased in popularity and adoption in the industry. With the improvement in technology, the research into DSS's has also increased. The abstract concept of a system that aids in the decision-making process has its roots in the late 1950's and early 1960's through research done at the Carnegie Institute of Technology. This later grew into information systems research through work done at the Massachusettes Institute of Technology in the 1960's. DSS's started to become a research topic of its own in the mid-1970's with more and more research following through the mid to late 1980's \cite{Keen1980, Power2003}.

\cite{Power2003} lists 5 main categories of DSS's which are,

\begin {itemize}
\item \textbf{Communication-driven DSS} is a type of DSS that emphasizes communications, collaboration and shared decision-making support. A simple bulletin board or threaded email is the most elementary level of functionality. The comp.groupware FAQ defines groupware as "software and hardware for shared interactive environments" intended to support and augment group activity. Groupware is a subset of a broader concept called Collaborative Computing. Communications-Driven DSS enable two or more people to communicate with each other, share information and co-ordinate their activities. Group Decision Support Systems or GDSS is a hybrid type of DSS that allows multiple users to work collaboratively in groupwork using various software tools. Examples of group support tools are: audio conferencing, bulletin boards and web-conferencing, document sharing, electronic mail, computer supported face-to-face meeting software, and interactive video.; 
\item \textbf{Data-driven DSS} is a type of DSS that emphasizes access to and manipulation of a time-series of internal company data and sometimes external data. Simple file systems accessed by query and retrieval tools provide the most elementary level of functionality. Data warehouse systems that allow the manipulation of data by computerized tools tailored to a specific task and setting or by more general tools and operators provide additional functionality. Data-driven DSS with On-line Analytical Processing (OLAP) provides the highest level of functionality and decision support that is linked to analysis of large collections of historical data. Executive Information Systems (EIS) and Geographic Information Systems (GIS) are special purpose Data-Driven DSS.;
\item \textbf{Document-driven DSS} is a relatively new field in Decision Support. A document driven DSS is focused on the retrieval and management of unstructured documents. Documents can take many forms, but can be broken down into three categories: Oral, written, and video. Examples of oral documents are conversations that are transcribed; video can be news clips, or television commercials; written documents can be written reports, catalogs, letters from customers, memos, and even e-mail.;
\item \textbf{Knowledge-driven DSS} can suggest or recommend actions to managers. These DSS are person-computer systems with specialized problem-solving expertise. The "expertise" consists of knowledge about a particular domain, understanding of problems within that domain, and "skill" at solving some of these problems.;
\item \textbf{Model-driven DSS} emphasize access to and manipulation of a model, for example, statistical, financial, optimization and/or simulation models. Simple statistical and analytical tools provide the most elementary level of functionality. Some OLAP systems that allow complex analysis of data may be classified as hybrid DSS systems providing both modeling and data retrieval and data summarization functionality. In general, model-driven DSS use complex financial, simulation, optimization or multi-criteria models to provide decision support. Model-driven DSS use data and parameters provided by decision makers to aid decision makers in analyzing a situation, but they are not usually data intensive, which means very large databases are usually not needed for a model-driven DSS.;
\end {itemize}

\cite{Power2003} also identifies 4 main characteristics of DSS's namely,

\begin {enumerate}
\item Inputs - these are what is used by the DSS analysis.;
\item User knowledge and expertise - this allows the system to decide how much it is relied on, and exactly what inputs must be analyzed with or without the user.;
\item Outputs - this is used so that the user of the system can analyze the data and potentially make a decision based on the information presented.;
\item Decision Making - this is the ultimate goal of the User who utilizes the DSS. The goal of the system is to aid the User in their decision-making process.;
\end {enumerate}

\newpage

\section{Transportation Information Systems of cities in other Developing Countries}

\paragraph{} So far in this chapter, we took a look at the literature related to the planning process of a public transportation company. We also reviewed some literature behind Decision Support Systems. Next we'll take a look at some Transportation Information Systems of cities in other Developing Countries.

During this research, data regarding information systems related to transportation services of 4 Indian cities, namely Bangalore, Mumbai, Delhi and Chennai, as well as the transportation services in Uganda (mainly Kampala) and the public bus transport service of Buenos Aires in Argentina was uncovered. Sri Lanka, India, Argentina and Uganda are all listed as Developing Countries according to the latest International Statistical Institute data \cite{ISI2013}.

Of the 4 Indian cities under consideration, Bangalore, Mumbai and Delhi had more modern Information Systems for their transport services. The Chennai system looked outdated and lacked proper maintenance. Common functionality that these 4 systems related to the indian cities provided was a Bus Route Finder (i.e.: when origin and destination stops are provided, the system provides the possible routes), Information Portal regarding Bus Routes and Bus Stops and information regarding hiring/renting a bus for private journeys. Furthermore, only the Bangalore and Chennai systems offered the functionality of commuter feedback \cite{BMTC1997, BEST1995, DTC2012, MTC2001}.

In Argentina, the public bus transport system is known as the "Colectivos" and although there is no proper information system for the bus service, private travel organizations have information portals regarding the service. One such portal is that which is run by Omnilineas, a licensed travel agency in Buenos Aires with a special focus on incoming tourism and bus travel. Their system provides Bus Route Information as well as information regarding booking tickets for long-distance trips \cite{Omnilineas2013}.