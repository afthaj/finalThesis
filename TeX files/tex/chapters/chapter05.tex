% begin Chapter ResearchEvaluation

\chapter {Research Evaluation}
\label{chapter-ResearchEvaluation}

\paragraph {} In the previous chapter, we took a look at the proposed system architecture and it's features. We also understood what the entities in the system were and how they relate to each other. In any research, the evaluation of the concept is as, if not more, important to the justifying the end result. Accordingly an evaluation of the prototype was carried out with the schedulers and the commuters.



\section{User Evaluation of Prototype}

\paragraph{} There were 3 stakeholders that used this research prototype. They are,

\begin{itemize}
\item Schedulers of the Transport Authority
\item Commuters
\item Bus Owners and Operators
\end{itemize}

\paragraph{} Of these 3 main types of users, the Schedulers act as the Admins and they can carry out the Admin functionalities listed under Section~\ref{systemFeatures}. Each of these users will have different views and functionalities associated with them. The User evaluation was carried out only with the Schedulers and the Commuters as I was unable to find any Bus Owners / Operators willing to participate in this study.

The prototype was hosted on a free hosting service and the Schedulers were interviewed while they were using the system to obtain a qualitative feedback on the prototype and its functionality. The URL of the public interface was sent to regular commuters to ascertain their feedback on the system. The prototype system can be accessed at \url{http://gaman.byethost4.com/}.


\subsection{Feedback of the Schedulers}

\paragraph{}Feedback from the schedulers were taken in a qualitative manner as there was a low number of schedulers in the authority. The inteeviews were recorded (with prior approval) and are available for review.

The response from the schedulers was overwhelmingly positive towards the system. They said they welcome a system like this and would gladly use it if it does eventually become a system ready for implementation. Furthermore, they mentioned that the User Interface was very simple, easy-to-use and intuitive.

The Schedulers also pointed out that an implementable system would need some added functionality such as the ability to store timetables, generate timetables automatically when given the data and the vehicle and crew scheduling functionalities. However, they said as this is a prototype system, the concept is correct, sound and suitable for further development.


\subsection{Feedback of the Commuters}

\paragraph{} Feedback from the commuters were gained in qualitative as well as quantitative form. After hosting the prototype online, a group of commuters were provided the link and given an explanation on how to use the system. Then they were asked to fill out a survey on their experience using the Gaman prototype system. The survey contained questions asking users to rate the various functionalities of the system, thus providing quantitative feedback regarding the system. The survey also contained questions that elicited comments from the commuters so that qualitative feedback was also gained.

The key points have been summarized and the results have been tabulated below. The Survey can be accessed by navigating to the home page of Gaman (provided in the previous chapter).

\begin {itemize}

\item Number of Participants: 39

\item Age group - Table~\ref{table-survey-ageGroupOfSurveyParticipants}

\begin{table} [H]
\centering
\begin{tabular}{|l|c|}
\hline
Age Group & Number of participants \\
\hline
20-35	&38 \\
10-19	&01 \\
\hline
\end{tabular}
\caption{Age Groups of Participants}
\label{table-survey-ageGroupOfSurveyParticipants}
\end{table}

\item Frequency of travel in the bus - Table~\ref{table-survey-frequencyOfTravelInTheBus}

\begin{table} [H]
\centering
\begin{tabular}{|l|c|}
\hline
Frequency of travel & Number of participants \\
\hline
Daily, one or two times a day	&20 \\
Daily, several times a day	&15 \\
A few times a week	&03 \\
A few times a month	&01 \\
\hline
\end{tabular}
\caption{Frequency of travel in the bus}
\label{table-survey-frequencyOfTravelInTheBus}
\end{table}

\end{itemize}

\subsection{Rating - Information Portal} 

\paragraph{} (Scale of 1 to 5 with 5 being "Very Useful" and 1 being "Not useful at all") - Table~\ref{table-survey-rating-InformationPortal} - Figure~\ref{image-ratingInformationPortal}

\begin{table} [H]
\centering
\begin{tabular}{|l|c|}
\hline
Rating & Number of participants \\
\hline
5	&09 \\
4	&20 \\
3	&07 \\
2	&03 \\
1	&00 \\
\hline
\end{tabular}
\caption{Rating - Information Portal}
\label{table-survey-rating-InformationPortal}
\end{table}

\begin {figure} [H]
\centering
\includegraphics [scale=0.6] {ratingInformationPortal}
\caption [Chart - Rating - Information Portal] {Chart - Rating - Information Portal}
\label {image-ratingInformationPortal}
\end {figure}

\paragraph{Qualitative comments received}
\begin {itemize}
\item "well i think infomation about bus routes is useful if we want to travel to a place that we dnt know using bus routes. most of the time i use google maps to navigate but it doesnt provide me the functionality of which bus route should i take. so i think this is a good effort"
\item "useful."
\item "It is a superb move. Very informative and useful. Hope this is mobile friendly as well."
\item "Very helpful. There are sufficient information regarding buses and bus routes and personals."
\item "It is great that we can use this system to get information about the buses and owners."
\item "Really good one"
\end {itemize}



\subsection{Rating - Bus Route Finder} 

\paragraph{} (Scale of 1 to 5 with 5 being "Very Useful" and 1 being "Not useful at all") - Table~\ref{table-survey-rating-BusRouteFinder} - Figure~\ref{image-ratingBusRouteFinder}

\begin{table} [H]
\centering
\begin{tabular}{|l|c|}
\hline
Rating & Number of participants \\
\hline
5	&12 \\
4	&16 \\
3	&09 \\
2	&02 \\
1	&01 \\
\hline
\end{tabular}
\caption{Rating - Bus Route Finder}
\label{table-survey-rating-BusRouteFinder}
\end{table}

\begin {figure} [H]
\centering
\includegraphics [scale=0.7] {ratingBusRouteFinder}
\caption [Chart - Rating - Bus Route Finder] {Chart - Rating - Bus Route Finder}
\label {image-ratingBusRouteFinder}
\end {figure}

\paragraph{Qualitative comments received}
\begin {itemize}
\item "simple but effective solution to easily find the routes."
\item "It's better if the information were presented in a better way, so that those can be found easily (E.g. provide searching facilities etc.)"
\item "More places need to be added to the system. It's better to give the bus route with the start and end destination as the first details. for some users sometimes giving only route number will not make sense very much."
\end {itemize}



\subsection{Rating - Feedback Functionality} 

\paragraph{} (Scale of 1 to 5 with 5 being "Very Useful" and 1 being "Not useful at all") - Table~\ref{table-survey-rating-FeedbackFunctionality} - Figure~\ref{image-ratingFeedbackFunctionality}

\begin{table} [H]
\centering
\begin{tabular}{|l|c|}
\hline
Rating & Number of participants \\
\hline
5	&09 \\
4	&14 \\
3	&13 \\
2	&03 \\
1	&00 \\
\hline
\end{tabular}
\caption{Rating - Feedback Functionality}
\label{table-survey-rating-FeedbackFunctionality}
\end{table}

\begin {figure} [H]
\centering
\includegraphics [scale=0.6] {ratingFeedbackFunctionality}
\caption [Chart - Rating - Feedback Functionality] {Chart - Rating - Feedback Functionality}
\label {image-ratingFeedbackFunctionality}
\end {figure}

\paragraph{Qualitative comments received}
\begin {itemize}
\item "Needs the ability to connect drivers and conductors to buses,feedbacks and complaints. drivers and conducters may get assigned to the buses by a time table component. ( couldn't test this since i couldn't see the needed form elements for filling those info while giving feedback )"
\item "Feedback to drivers and conductors may not reach them because of the low computer literacy. However, if there is any responsible person to handle or communicate with them then it is really useful function"
\end {itemize}



\subsection{Rating - Comlaints Functionality}

\paragraph{} (Scale of 1 to 5 with 5 being "Very Useful" and 1 being "Not useful at all") - Table~\ref{table-survey-rating-ComplaintsFunctionality} - Figure~\ref{image-ratingComplaintsFunctionality}

\begin{table} [H]
\centering
\begin{tabular}{|l|c|}
\hline
Rating & Number of participants \\
\hline
5	&08 \\
4	&16 \\
3	&10 \\
2	&05 \\
1	&00 \\
\hline
\end{tabular}
\caption{Rating - Complaints Functionality}
\label{table-survey-rating-ComplaintsFunctionality}
\end{table}

\begin {figure} [H]
\centering
\includegraphics [scale=0.6] {ratingComplaintsFunctionality}
\caption [Chart - Rating - Complaints Functionality] {Chart - Rating - Complaints Functionality}
\label {image-ratingComplaintsFunctionality}
\end {figure}

\paragraph{Qualitative comments received}
\begin {itemize}
\item "This is a very useful functionality that would help in increasing and maintaining the quality of the bus services."
\item "This is really good because sometimes we have to bare all the sh***y experiences or will be able to share it with few friends and forget about those. But this functionality facilitates us to share our good/bad experiences with others and even will be able to do something."
\item "It's very useful to have such functionality, because that is one of  the main problem we can see in the public transport system in Sri Lanka."
\item "I wish the complain data were more specific. For example, user might be interested in knowing the bus-route associated with a certain complain or he might be interested viewing all the complains related to a certain but-route."
\item "If it functioned properly this will have a big impact. Simply a user can share their own experience. But there are possibilities of fake complaints."
\end {itemize}



\subsection{Rating - User Interface}

\paragraph{} (Scale of 1 to 5 with 5 being "Very easy to use" and 1 being "Very difficult to use") - Table~\ref{table-survey-rating-UserInterface} - Figure~\ref{image-ratingUserInterface}

\begin{table} [H]
\centering
\begin{tabular}{|l|c|}
\hline
Rating & Number of participants \\
\hline
5	&16 \\
4	&11 \\
3	&09 \\
2	&02 \\
1	&01 \\
\hline
\end{tabular}
\caption{Rating - User Interface}
\label{table-survey-rating-UserInterface}
\end{table}

\begin {figure} [H]
\centering
\includegraphics [scale=0.55] {ratingUserInterface}
\caption [Chart - Rating - User Interface] {Chart - Rating - User Interface}
\label {image-ratingUserInterface}
\end {figure}

\paragraph{Qualitative comments received}
\begin {itemize}
\item "good. but can improve. trade off between simplicity and ease of use."
\item "Simple and nice! good work!"
\item "Simple, but effective and attractful. Nicely developed."
\item "Looks a bit too linear in my opinion... more vibrant colours should be used to highlight the important and most frequently used features.."
\item "I like the simplest interface and colors. Some pages have textboxes whichs seems very small comparing to the screen size. Better if you add some autoresize them"
\end {itemize}



\subsection{Survey Question}

\paragraph{Would you use a system like this if it was implemented?} Table~\ref{table-survey-question-WouldYouUseASystemLikeThis} - Figure~\ref{image-questionWouldYouUse}

\begin{table} [H]
\centering
\begin{tabular}{|l|c|}
\hline
Rating & Number of participants \\
\hline
Oh my god, yes!	&16 \\
Yes	&17 \\
No	&06 \\
\hline
\end{tabular}
\caption{Survey Question - Would you use a system like this if it was implemented?}
\label{table-survey-question-WouldYouUseASystemLikeThis}
\end{table}

\begin {figure} [H]
\centering
\includegraphics [scale=0.55] {questionWouldYouUse}
\caption [Chart - Survey Question] {Chart - Survey Question}
\label {image-questionWouldYouUse}
\end {figure}






